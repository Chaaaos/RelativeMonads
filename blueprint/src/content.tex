% In this file you should put the actual content of the blueprint.
% It will be used both by the web and the print version.
% It should *not* include the \begin{document}
%
% If you want to split the blueprint content into several files then
% the current file can be a simple sequence of \input. Otherwise It
% can start with a \section or \chapter for instance.


% [TODO : ]

% \href{link} to add an hyperref.

% \begin{equation*} [equation] \end{equation*}

%\begin{definition}[NomDefinition] \label{def:label-def} \lean{LeanDefinition \leanok [definition] \end{definition}

% \begin{proposition} \label{prop:label-prop} \leanok \lean{LeanProp} \uses{def:definition-1, def:definition-2} [proposition] \end{proposition}

% \begin{proof} \leanok [proof] \end{proof}


\begin{abstract}
  The following document is intended as an outline for the formalisation of \emph{Relative Monads} using the proof assistant \emph{Lean 4}. Part of this formalisation consists in trying to establish relations between relative monads, monads and Kleisli triples.
\end{abstract}

% \section{Introduction}


\chapter{Relative Monads}
\section{Basic Definitions}
\begin{definition}[Relative Monads]
  %\label{def:rel-mon}
  %\lean{LeanDefinition}
  %\leanok
  Given two categories $\C$ and $\D$ and a functor $J : \C \to \D$ \emph{(root)}, a relative monad on $\D$ over $\C$ consists in :
  \begin{itemize}
  \item A function $T : \C \to \D$ ;
  \item A natural transformation $\eta : J \to T$ \emph{(unit)} ;
    \item For each morphism in $\D$ of the form $f : J a \to T b$ (where $a$ and $b$ are objects of $\C$), a morphism $\lift{f} : T a \to T b$ \emph{(extension /lifting operation)} ;
  \end{itemize}
  satisfying the following conditions :
  \begin{itemize}
  \item $\forall a \in obj_{\C}, (f : J a \to T a), f = \lift{f} \circ \eta_a$
    \emph{(left unitality)}
  \item $\forall a \in obj_{\C}, \lift{(\eta_a)} = \mathbb{1}_{T a}$
    \emph{(right unitality)}
  \item $\forall a, b, c \in obj_{\C}, $\\
    $(f : J a \to T a), (g : J b \to T c),$\\
    $\lift{\lift{g}\circ f} = \lift{g} \circ \lift{f} $
    
  \end{itemize}

\begin{proposition}
  %\label{prop:func-nat}
  %\lean{LeanProp}
  %\leanok
  It follows from the conditions in the definition of a relative monad that :
  \begin{itemize}
  \item $T$ defines a \emph{functor} ;
    \item $\eta$ and $\lift{·}$ are \emph{natural transformations}.
    \end{itemize}
\end{proposition}
  
\end{definition}

\begin{definition}[Relative Monads morphisms]
  %\label{def:rel-mon-morphism}
  %\lean{LeanDefinition}
  %\leanok
  %[TODO : definition]
\end{definition}

\begin{proposition}[Category of Relative monads]
  %\label{prop:rel-mon-cat}
  %\lean{LeanProp}
  %\leanok
  %[TODO : proposition]
\end{proposition}

\begin{proof}
  %\leanok
  %[TODO : proof]
\end{proof}


\section{Examples of Relative Monads}

\subsection*{$T = J$}
% [TODO : find a better name.]

\chapter{External presentation of monads - Kleisli Triples}

\begin{definition}[Kleisli Triples]
  %\label{def:kleisli-triples}
  %\lean{LeanDefinition}
  %\leanok
  %[TODO : definition]
\end{definition}

\begin{definition}[Morphisms of Kleisli triples]
  %\label{def:kleisli-morphisms}
  %\lean{LeanDefinition}
  %\leanok
  %[TODO : definition]
\end{definition}

  \begin{proposition} \label{prop:kleisli-triple-cat}
    %\leanok
    %\lean{LeanProp}
    %\uses{def:definition-1, def:definition-2}
    "Given a category $\C$, the Kleisli triples on $\C$ and the morphisms between them form a category."
    %[TODO : Rephrase]
  \end{proposition}
  \begin{proof}
  %\leanok
  %[TODO : proof]
\end{proof}

  \begin{proposition} \label{prop:injection-kt-rm}
    %\leanok
    %\lean{LeanProp}
    %\uses{def:definition-1, def:definition-2}
    "Given a category $\C$, each Kleisli triple on $\C$ corresponds to a relative monad on $\C$ over itself." %[TODO : Rephrase...]
      % embedding ?
  \end{proposition}
  \begin{proof}
  %\leanok
  %[TODO : proof]
\end{proof}



\chapter{Kleisli triples are equivalent to Monads}

\begin{proposition}
  %\label{prop:kt-equiv-monads}
  %\lean{LeanProp}
  %\leanok
  "Given a category $\C$, Kleisli triples on $\C$ are equivalent to monads on $\C$."
\end{proposition}
\begin{proof}
  %\leanok
  %[TODO : proof]
\end{proof}

\begin{proposition}
  %\label{prop:kt-equiv-monads}
  %\lean{LeanProp}
  %\leanok
  "Given a category $\C$, each monad on $\C$ corresponds to a relative monad on $\C$ over itself."
\end{proposition}
\begin{proof}
  %\leanok
  %[TODO : proof]
\end{proof}

  % Morita equivalence (the category of Kleisli triples is equivalent to the category of monads) ;
 %All monads are relative monads. 

% (Add a section about "programming language monads" ?)
